% XeLaTeX can use any Mac OS X font. See the setromanfont command below.
% Input to XeLaTeX is full Unicode, so Unicode characters can be typed directly into the source.

% The next lines tell TeXShop to typeset with xelatex, and to open and save the source with Unicode encoding.

%!TEX TS-program = xelatex
%!TEX encoding = UTF-8 Unicode

\documentclass[12pt]{article}


\usepackage{geometry}                % See geometry.pdf to learn the layout options. There are lots.
\geometry{letterpaper}                   % ... or a4paper or a5paper or ... 
%\geometry{landscape}                % Activate for for rotated page geometry
%\usepackage[parfill]{parskip}    % Activate to begin paragraphs with an empty line rather than an indent
\usepackage{graphicx}
\usepackage{amssymb}
\usepackage{amsmath}
\usepackage{xeCJK}
\usepackage{ctex}
\usepackage[center]{titlesec}
%\usepackage[compress]{natbib}
\usepackage[super,square,compress]{natbib}
%\newcommand{\ucite}[1]{\textsuperscript{\cite{#1}}}%上标用\ucite{ };文中用\cite{ }
%\newcommand{\ucitet}[1]{\textsuperscript{\citet{#1}}}
\setCJKmainfont{华文宋体}
% Will Robertson's fontspec.sty can be used to simplify font choices.
% To experiment, open /Applications/Font Book to examine the fonts provided on Mac OS X,
% and change "Hoefler Text" to any of these choices.
\usepackage[colorlinks=true,citecolor=blue,linkcolor=blue]{hyperref}
\usepackage{fontspec,xltxtra,xunicode}
\defaultfontfeatures{Mapping=tex-text}
\setromanfont[Mapping=tex-text]{Times New Roman}%Hoefler Text}
\setsansfont[Scale=MatchLowercase,Mapping=tex-text]{Times New Roman}
\setmonofont[Scale=MatchLowercase]{Andale Mono}
%_______________________________________________________________






\title{利用VLBI差分时延反演嫦娥三号奔月姿态\\ }
\author{zzb \\ }

%\date{}                                           % Activate to display a given date or no date
%\date{}      
%2015,12,31
\begin{document}
\maketitle

\section{研究计划}
分析研究意义:进一步提高VLBI定轨精度;国外做的怎么样?

先分析时延和时延率的整体情况;

明确哪一段时延好,哪一段时延率好;

时延率误差大的原因是什么?

介质改正的精度如何?差分用不着?


进而选取合适的弧段,要求各条基线变化稳定


\section{理论推导}

在星基坐标系中,设星载天线$A_1$、$A_2$的相位中心相对天线质心的坐标分别为$X_1$、$X_2$,
设地面VLBI站两天线分别为$S_1$、$S_2$,

从地面VLBI站到



L在基线方向的投影

估计的参数包括:

三个旋转角度,

由于基线长度已知,因此可以估计L


时延率信息怎么用?


\section{数据选取}
看差分时延的弥散度与精度



判断是站坐标引起的,还是自旋引起的

站坐标引起的话应该与某些基线有关,如果是卫星姿态引起的则可以就数据进行进一步分析






%
%




\end{document}  


%
%误差补偿前
%
%63.366248 ± 0.000639
%-12.392477 ± 0.000586
%8.695748 ± 0.000140
%-0.004467 ± 0.000473
%19.191258 ± 10.323624
%-11.233190 ± 10.715057
%-26.902845 ± 8.483806
%38.118444 ± 0.008031
%
%\({\chi ^2}\)=3.250
%
%补偿后
%
%63.366474 ± 0.000569
%
%-12.392889 ± 0.000542
%
%8.695305 ± 0.000159
%
%-0.003171 ± 0.000406
%
%18.728830 ± 7.720389
%
%-11.875081 ± 8.002122
%
%-27.128993 ± 6.307083
%
%38.118568 ± 0.005973
%
%1.534
%
%
%63.366432 ± 0.000533
%-12.393171 ± 0.000513
%8.695164 ± 0.000171
%-0.002779 ± 0.000376
%18.639219 ± 6.900655
%-11.997838 ± 7.143747
%-27.200913 ± 5.608665
%38.118615 ± 0.005312
%1.024
%
%
%0.00037463;%+0.00018977+2.4979e-5;
%
%63.366426 ± 0.000529
%-12.393201 ± 0.000510
%8.695150 ± 0.000173
%-0.002743 ± 0.000373
%18.631017 ± 6.825523
%-12.006118 ± 7.064843
%-27.208050 ± 5.543926
%38.118620 ± 0.005251
%0.976
%
%\({\chi ^2}\)  




%\begin{verbatim}
%dof=229;
%chi20=sg2;
%sgs=0.0001;
%dsgs=0;
%
%
%for i=1:30
%    sgs=sgs+dsgs;
%    sgs2=sgs.^2;
%    chi2=(1/dof)*V'*inv(QLL+diag(ones(1,size(QLL,1))*sgs2))*V;
%    %sg2=chi20
%    
%    pFpsgs=-4*(chi2-1)*(1/dof)*sgs*V'*inv((QLL+diag(ones(1,size(QLL,1))*sgs2)).^2)*V;
%    
%    dsgs=(-(chi2-1).^2)/pFpsgs;
%    disp(num2str(sgs))
%    if sgs<1e-7
%        break
%    end 
%end
%
%\end{verbatim}